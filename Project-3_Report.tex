\documentclass[twoside]{report}

\title{Project 3}
\date{31 January 2020}
\author{Michael Hrishenko, \LaTeX\ WIZARD}
%LVLS: NEWBIE,JOURNEYMAN,DAEMON,WIZARD,WHEEL,SUPERUSER,STAFF,ADMIN,SYSLORD,HAKMEM,DEMIGOD,UBERGEEK,DARPA,TURING,FORBIN,NUCLEAR,L33T,G-MAN,OMICRON,M.I.B.,Q DIVISION,NSAMRSOC,QUANTUM,ORBIT,TRIAD,COSMOS,ASTRAL,STELLAR,ULTRA,LUNA,COSMIC,MAJESTIC,META

\usepackage{lipsum}
\usepackage{enumerate}
\usepackage{textcomp}
\usepackage{amsmath}
\usepackage{hyperref}
\usepackage{listings}
\usepackage{634format}
\usepackage{holtex}
\usepackage{holtexbasic}
%\input{commands}

\begin{document}
\lstset{language=ML}
\maketitle
\begin{abstract}
This report begins actual proofs and theorems in HOL and ML, using logic structures and principles to achieve desired ends. Several new tools including Holmake and EmiTex are introduced. Our goal is to show how ML code and HOL theorems are reversible and reproducible in order to create assured code by design.
\end{abstract}

\textbf{Acknowledgements}: Professors Marvine Hamner, Shiu-Kai Chin, \& Susan Older

\tableofcontents

\chapter{Executive Summary}
\label{cha:executive-summary}
Unable to complete problems 8.4.2 \& 8.4.3 due to time constraints. Problem 8.4.1 completed but unable to compile properly given Holmake errors:
\begin{scriptsize}
    \begin{verbatim}
% Holmake                                                                     4:22:11
chapter8Theory                                        real:    0s  user:    0sFAIL<1>
 Couldn't find required output file: /home/troy/dev/ms500/CIS-634_AsrFound/03_Project/chapter8Theory.sml
     \end{verbatim}
\end{scriptsize}
Continuing to troubleshoot Holmake errors, however already wasted enough time and focusing now on completing problems. All problems from Chapter 7 completed successfully, no errors to report. Many thanks to Professor Hamner for helping me work around my ever-changing schedule.


\chapter{Exercise 7.3.1}
\label{cha:7-3-1}
\section{Problem Statement}
\begin{enumerate}
    \item In ML, define functions (and test them on examples) corresponding to each function below. For each of the functions, you will define two ML functions,
    \begin{enumerate}
        \item the first using fn and val to define and name the function, and
        \item the other using fun to define and name the function.
        \begin{enumerate}
            \item A function that takes a 3-tuple of integers (x;y;z) as input and returns the value corresponding to the sum x+y+z.
            \item A function that takes two integer inputs x and y (where x is supplied first followed by y) and returns the boolean value corresponding to x < y.
        \end{enumerate}
    \end{enumerate}
    \item Make sure you use pattern matching. For example, suppose the function is lx:(ly:2x+y). We would define in ML 1. val funEx1 = (fn x => (fn y => 2*x + y)), and 2. fun funEx2 x y = 2*x + y As a naming convention, use the names funA1, funA2, funB1, funB2, etc.
\end{enumerate}
\section{Relevant Code}
    \begin{lstlisting}[frame=trBL]
(*** A ***)
val funA1 = (fn x => (fn y => (fn z => x + y + z)));
fun funA2 x y z = x + y + z;
(*** B ***)
val funB1 = (fn y => (fn x => x < y));
fun funB2 x y = x < y;
    \end{lstlisting}
\section{Test Cases}
%\setcounter{sessioncount}{0}
    \begin{scriptsize}
    \begin{verbatim}
(*** A ***)
funA1 1 1 1;
funA2 1 1 1;
(*** B ***)
funB1 1 2;
funB2 1 2;
    \end{verbatim}
    \end{scriptsize}
\section{Execution Transcripts}
\begin{scriptsize}
    \begin{verbatim}
# val it = 3: int
> val it = 3: int
> # val it = false: bool
    \end{verbatim}
\end{scriptsize}

\appendix{}
\chapter{Source Code for 7.3.1}
\label{cha:apdx-a}
%\lstinputlisting{ML/ex-4-6-3.sml}

\end{document}